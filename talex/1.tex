\begin{array}{cols}
高等数学上册练习题
一、选择题
1.  x \rightarrow 0  时, 下列函数是  x  的()阶无穷小.
A.  \ln (1+x) 
B.  e^{x}-1 
C.  e^{x^{3}}-1 
D.  \sqrt{1+x^{2}}-1 
E.  1-\cos x 
F.  \arctan x 
G.  \ln \left(1+x^{2}\right) 
2. 设函数  f(x)=\left\{\begin{array}{ll}x^{-3}, 0<x \leq 1, \\ 3 x, & x>1,\end{array}\right.  则  x=1  是  f(x)  的 ( ).
A. 连续点
B. 可去间断点
C. 跳跃间断点
D. 第二类间断点
3. 设函数  f(x)=\left\{\begin{array}{l}2^{x}+3,0<x \leq 1, \\ 3 x+4, x \leq 0,\end{array}\right.  则  x=0  是  f(x)  的 ( ).
A. 连续点
B. 第二类间断点
C. 可去间断点
D. 跳跃间断点
4. 函数  f(x)  在点  x=x_{0}  处可导, 且  f^{\prime}\left(x_{0}\right)=2 , 则  \lim _{h \rightarrow 0} \frac{f\left(x_{0}+2 h\right)-f\left(x_{0}\right)}{6 h}=(\quad) .
A.  \frac{2}{3} 
B.  -\frac{2}{3} 
C.  \frac{3}{2} 
D.  -\frac{3}{2} 
5. 函数  f(x)  在点  x=x_{0}  处可导, 且  f^{\prime}\left(x_{0}\right)=\frac{1}{2} , 则  \lim _{h \rightarrow 0} \frac{f\left(x_{0}\right)-f\left(x_{0}-3 h\right)}{h}=(\quad) .
A.  \frac{2}{3} 
B.  -\frac{2}{3} 
C.  \frac{3}{2} 
D.  -\frac{3}{2} 
6. 下列广义积分收敛的是()。
A.  \int_{e}^{+\infty} \frac{\ln x}{x} d x 
B.  \int_{e}^{+\infty} \frac{1}{x \ln ^{2} x} d x 
C.  \int_{e}^{+\infty} \frac{\sqrt{\ln x}}{x} d x 
D.  \int_{e}^{+\infty} \frac{1}{x \sqrt{\ln x}} d x 
7. 下列广义积分收敛的是().
A.  \int_{e}^{+\infty} \frac{\ln x}{x} d x 
B.  \int_{e}^{+\infty} \frac{1}{x \ln ^{2} x} d x 
C.  \int_{e}^{+\infty} \frac{\sqrt{\ln x}}{x} d x 
D.  \int_{e}^{+\infty} \frac{1}{x \sqrt{\ln x}} d x 
8. 定积分  \int_{b}^{a} f^{\prime}(2 x) d x=(\quad) .
A.  f(a)-f(b) 
B.  f(2 a)-f(2 b) 
C.  \frac{1}{2}[f(2 a)-f(2 b)] 
D.  2[f(2 a)-f(2 b)] 
9. 下列数列收敛的是()
A.  \sin n 
B.  (-1)^{n}+\frac{1}{n} 
C.  \left(1+\frac{1}{n}\right)^{n} 
\end{array}
